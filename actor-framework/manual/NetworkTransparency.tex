\section{Network Transparency}

All actor operations as well as sending messages are network transparent.
Remote actors are represented by actor proxies that forward all messages.
All functions shown in this section can be accessed by including the header \lstinline^"caf/io/all.hpp"^ and live in the namespace \lstinline^caf::io^.

\subsection{Publishing of Actors}

\begin{lstlisting}
void publish(actor whom, std::uint16_t port, const char* addr = 0)
\end{lstlisting}

The function \lstinline^publish^ binds an actor to a given port.
It throws \lstinline^network_error^ if socket related errors occur or \lstinline^bind_failure^ if the specified port is already in use.
The optional \lstinline^addr^ parameter can be used to listen only to the given IP address.
Otherwise, the actor accepts all incoming connections (\lstinline^INADDR_ANY^).

\begin{lstlisting}
io::publish(self, 4242);
self->become (
  on(atom("ping"), arg_match) >> [](int i) {
    return make_message(atom("pong"), i);
  }
);
\end{lstlisting}

\subsection{Connecting to Remote Actors}

\begin{lstlisting}
actor remote_actor(const char* host, std::uint16_t port)
\end{lstlisting}

The function \lstinline^remote_actor^ connects to the actor at given host and port.
A \lstinline^network_error^ is thrown if the connection failed.

\begin{lstlisting}
auto pong = remote_actor("localhost", 4242);
self->send(pong, atom("ping"), 0);
self->become (
  on(atom("pong"), 10) >> [=] {
    self->quit();
  },
  on(atom("pong"), arg_match) >> [=](int i) {
    return make_message(atom("ping"), i+1);
  }
);
\end{lstlisting}
